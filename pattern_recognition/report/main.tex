%!TEX TS-program = xelatex
% !TeX program = xelatex
%!TEX encoding = UTF-8 Unicode
%----------------------------------------------------------------------------------------
%   Доорх хэсгийг өөрчлөх шаардлагагүй
%----------------------------------------------------------------------------------------
\documentclass[12pt,A4]{report}

\usepackage{fontspec,xltxtra,xunicode}
\setmainfont[Ligatures=TeX]{Times New Roman}
\setsansfont{Arial}

% \usepackage[utf8x]{inputenc}
% \usepackage[mongolian]{babel}
%\usepackage{natbib}
\usepackage{geometry}
%\usepackage{fancyheadings} fancyheadings is obsolete: replaced by fancyhdr. JL
\usepackage{fancyhdr}
\usepackage{float}
\usepackage{afterpage}
\usepackage{graphicx}
\usepackage{amsmath,amssymb,amsbsy}
\usepackage{dcolumn,array}
\usepackage{tocloft}
\usepackage{dics}
\usepackage{nomencl}
\usepackage{upgreek}
\newcommand{\argmin}{\arg\!\min}
\usepackage{mathtools}
\usepackage[hidelinks]{hyperref}

\usepackage{algorithm}
\usepackage{algpseudocode}

\usepackage{listings}
\DeclarePairedDelimiter\abs{\lvert}{\rvert}%
\makeatletter
\usepackage{caption}
\captionsetup[table]{belowskip=0.5pt}
\usepackage{subfiles}

\usepackage{listings}

\usepackage{color}
\definecolor{lightgray}{rgb}{.9,.9,.9}
\definecolor{darkgray}{rgb}{.4,.4,.4}
\definecolor{purple}{rgb}{0.65, 0.12, 0.82}

\lstdefinelanguage{JavaScript}{
  keywords={typeof, new, true, false, catch, function, return, null, catch, switch, var, if, in, while, do, else, case, break},
  keywordstyle=\color{blue}\bfseries,
  ndkeywords={class, export, boolean, throw, implements, import, this},
  ndkeywordstyle=\color{darkgray}\bfseries,
  identifierstyle=\color{black},
  sensitive=false,
  comment=[l]{//},
  morecomment=[s]{/*}{*/},
  commentstyle=\color{purple}\ttfamily,
  stringstyle=\color{red}\ttfamily,
  morestring=[b]',
  morestring=[b]"
}

\lstset{
   language=JavaScript,
   backgroundcolor=\color{lightgray},
   extendedchars=true,
   basicstyle=\footnotesize\ttfamily,
   showstringspaces=false,
   showspaces=false,
   numbers=left,
   numberstyle=\footnotesize,
   numbersep=9pt,
   tabsize=2,
   breaklines=true,
   showtabs=false,
   captionpos=b
}
\renewcommand{\lstlistingname}{Код}
\renewcommand{\lstlistlistingname}{\lstlistingname ын жагсаалт}

\usepackage{color}
\definecolor{codegreen}{rgb}{0,0.6,0}
\definecolor{codegray}{rgb}{0.5,0.5,0.5}
\definecolor{codepurple}{rgb}{0.58,0,0.82}
\definecolor{backcolour}{rgb}{0.99,0.99,0.99}
 
\lstdefinestyle{mystyle}{
    basicstyle=\ttfamily\small,
    backgroundcolor=\color{backcolour},   
    commentstyle=\color{codegreen},
    keywordstyle=\color{magenta},
    numberstyle=\tiny\color{codegray},
    stringstyle=\color{codepurple},
    %basicstyle=\footnotesize,
    breakatwhitespace=false,         
    breaklines=true,                 
    captionpos=b,                    
    keepspaces=false,                 
    numbers=left,                    
    numbersep=10pt,                  
    showspaces=false,                
    showstringspaces=true,
    showtabs=false,                  
    tabsize=2
}
 
\lstset{style=mystyle, label=DescriptiveLabel} 

\let\oldabs\abs
\def\abs{\@ifstar{\oldabs}{\oldabs*}}
\makenomenclature
\begin{document}


%----------------------------------------------------------------------------------------
%   Өөрийн мэдээллээ оруулах хэсэг
%----------------------------------------------------------------------------------------

% Дипломийн ажлын сэдэв
\title{Хэв танилт хичээлийн лабораторийн тайлан}
% Дипломын ажлын англи нэр
\titleEng{Laboratory report for the pattern recognition}
% Өөрийн овог нэрийг бүтнээр нь бичнэ
\author{Даянгийн Балжинням}
% Өөрийн овгийн эхний үсэг нэрээ бичнэ
\authorShort{Д. Балжинням}
% Удирдагчийн зэрэг цол овгийн эхний үсэг нэр
\supervisor{Б. Сувдаа}
% Хамтарсан удирдагчийн зэрэг цол овгийн эхний үсэг нэр
\cosupervisor{ }

% СиСи дугаар 
\sisiId{20B1NUM0563}
% Их сургуулийн нэр
\university{МОНГОЛ УЛСЫН ИХ СУРГУУЛЬ}
% Бүрэлдэхүүн сургуулийн нэр
\faculty{ХЭРЭГЛЭЭНИЙ ШИНЖЛЭХ УХААН, ИНЖЕНЕРЧЛЭЛИЙН СУРГУУЛЬ}
% Тэнхимийн нэр
\department{МЭДЭЭЛЭЛ, КОМПЬЮТЕРИЙН УХААНЫ ТЭНХИМ}
% Зэргийн нэр
\degreeName{Тайлан}
% Суралцаж буй хөтөлбөрийн нэр
\programeName{Програм хангамж(D061302)}
% Хэвлэгдсэн газар
\cityName{Улаанбаатар}
% Хэвлэгдсэн огноо
\gradyear{2023 оны 9 сар}


%----------------------------------------------------------------------------------------
%   Доорх хэсгийг өөрчлөх шаардлагагүй
%----------------------------------------------------------------------------------------
\include{src/main-pre}

% Удиртгалыг оруулж ирэх ба abstract.tex файлд удиртгалаа бичнэ
\begin{abstract}
	Энэхүү тайланд дүрсийг таних талбарт олон тооны зургуудаас дундаж зургийг тооцоолж, зураг боловсруулах техникийн гүнзгий дүн шинжилгээг өгдөг. Энэ ажлын хүрээнд зураг унших, зурагний хэмжээг өөрчилж матрицыг нь адил болгох, дундаж зураг тооцолох, тухайн зургийн дундаж зурагнаас хэр ялгаатайг олох зэргийг үзсэн.

	Хэв танилт нь машин сургалт, зураг боловсруулалт зэрэг янз бүрийн салбарт чухал үүрэг гүйцэтгэдэг. Энэ нь өгөгдлийн хэв маяг, зүй тогтлыг олоход тусладаг. Энэхүү тайлангийн гол зорилго нь зурагний бусад зурагнаас хэрхэн өөр байгааг, дундаж зураг яаж олох, дижитал байдлаар зураг хэрхэн хадгалагддаг зэргийг судлах зорилготой.
\end{abstract}


%----------------------------------------------------------------------------------------
%   Дипломын үндсэн хэсэг эндээс эхэлнэ
%----------------------------------------------------------------------------------------
%\addcontentsline{toc}{part}{БҮЛГҮҮД}
% Шинэ бүлэг
% \chapter{Тайлан боловсруулах зөвлөмж}
% \subfile{writing.tex}
\subfile{src/body.tex}

\subfile{src/bibliography.tex}

\end{document}
