\begin{abstract}
	Энэхүү тайланд дүрсийг таних талбарт олон тооны зургуудаас дундаж зургийг тооцоолж, зураг боловсруулах техникийн гүнзгий дүн шинжилгээг өгдөг. Энэ ажлын хүрээнд зураг унших, зурагний хэмжээг өөрчилж матрицыг нь адил болгох, дундаж зураг тооцолох, тухайн зургийн дундаж зурагнаас хэр ялгаатайг олох зэргийг үзсэн.

	Хэв танилт нь машин сургалт, зураг боловсруулалт зэрэг янз бүрийн салбарт чухал үүрэг гүйцэтгэдэг. Энэ нь өгөгдлийн хэв маяг, зүй тогтлыг олоход тусладаг. Энэхүү тайлангийн гол зорилго нь зурагний бусад зурагнаас хэрхэн өөр байгааг, дундаж зураг яаж олох, дижитал байдлаар зураг хэрхэн хадгалагддаг зэргийг судлах зорилготой.
\end{abstract}
