\chapter{Дүрс боловсруулах /Дунд хэсэг/}

\section{Histogram Processing}
\subsection{Histogram Processing}
Гистограмын боловсруулалт гэдэг нь түүний гистограммын тархалтыг тохируулах замаар
зургийн тодосгогчийг (Contrast) сайжруулахын тулд дүрс боловсруулахад ашигладаг арга юм.
\textbf{\(X\)} тэнхлэг нь боломжит эрчмийн түвшин бүрийг, \textbf{\(Y\)} тэнхлэг
нь харгалзах эрчимтэй пикселийн тоог илэрхийлдэг.

Гистограмм боловсруулах техникт гистограмм сунгах (Contrast stretching),
гистограмыг гулсуулах (Brightness modification), гистограмм тэгшитгэх (Histogram equalization)
зэрэг орно. Эдгээр техникүүд нь дүрсний харагдах байдлыг
сайжруулахын тулд зургийн гистограммыг өөрчилдөг. Гистограммын
боловсруулалтын зорилго нь ихэвчлэн жигд тархсан гистограммыг олж авах бөгөөд бүх
эрчим хүчний түвшинг ижил хэмжээгээр ашигладаг\cite{histogram}.

\subsection{Монотон функц}
Математикт функцийг тухайн муж дахь бүх x ба y-ийн хувьд, хэрэв
x ≤ y бол f\(x\) ≤ f\(y\) бол монотон өсөх (эсвэл энгийнээр хэлбэл монотон) гэж
нэрлэдэг. Энэ нь функц нь хаана ч буурахгүй гэсэн үг юм. Өөрөөр хэлбэл,
оролтын утгууд өсөхөд гаралтын утгууд буурахгүй.

Монотон функцууд нь тэдгээрийн графикууд нь локал
максимум эсвэл минимумгүй байх шинж чанартай байдаг. Эдгээр
нь энгийн, урьдчилан таамаглаж болохуйц тул математик
болон янз бүрийн салбарт хэрэгтэй\cite{monoton}.

\begin{figure}
    \centering
    \includegraphics[scale=0.7]{src/img/dasgal6.png}
    \caption{Histogram}
\end{figure}
\begin{figure}
    \centering
    \includegraphics[scale=0.7]{src/img/output.png}
    \caption{Histogram equalized}
\end{figure}

\newpage
\section{Local Enhancement}
Local Enhancement нь зургийн чанарыг сайжруулахын тулд зураг боловсруулахад
ашигладаг техник юм. Зургийг бүхэлд нь биш харин жижиг хэсгүүдийн
пикселүүдийг өөрчилснөөр зургийн contrast-г сайжруулдаг тул үүнийг Local Enhancement гэж нэрлэдэг.

\begin{enumerate}
    \item Зураг хуваах: Зургийг жижиг, давхцахгүй хавтан болгон хуваана.
    \item Гистограмыг тооцоолох: Хавтан тус бүрийн пикселийн эрчимжилтийн тархалтыг тооцоол.
    \item Ялгааг хязгаарлах: Noise нэмэгдүүлэхээс сэргийлэхийн тулд гистограммыг тодорхой утгаар таслана.
    \item Гистограммыг тэгшитгэх: Ялгааг сайжруулахын тулд хавтан бүрт пикселийн эрчмийг тараана.
    \item Интерполяци: Хоёр шугаман интерполяцыг ашиглан хавтангийн хилийг тэгшлэнэ.
    \item Нэгтгэх: Хавтангуудыг хооронд нь холбож, эцсийн, сайжруулсан дүрсийг бүрдүүлнэ.
\end{enumerate}


\begin{lstlisting}[language=Python, caption=Local Enhancement, frame=single]
import cv2
import numpy as np
import matplotlib.pyplot as plt

img = cv2.imread('chap3/Fig0326(a)(embedded_square_noisy_512).tif', 0)

clahe: cv2.CLAHE = cv2.createCLAHE(clipLimit=2.0, tileGridSize=(3,3))
enhanced = clahe.apply(img)
equalization = np.uint8(enhanced * 250) # type: ignore

fig, axs = plt.subplots(2, 2, figsize=(10, 10))

axs[0, 0].imshow(img, cmap='gray')
axs[0, 1].imshow(enhanced, cmap='gray')
axs[1, 0].imshow(equalization, cmap='gray')
\end{lstlisting}


\begin{figure}[h!]
    \centering
    \includegraphics[scale=0.3]{src/img/daalgavar7.png}
    \caption{Local Enhancement}
\end{figure}

\section[OR/AND]{Enhancement using Arithmetic / Logic Operations}
% Зураг нь ардаа битүүдээр дүрслэгдсэн байгаа ба
\begin{figure}[h!]
    \centering
    \includegraphics[scale=0.7]{src/img/dasgal8.png}
    \caption{Arithmetic / Logic Operations}
\end{figure}

\begin{figure}[h!]
    \centering
    \includegraphics[scale=0.3]{src/img/dasgal9.png}
    \caption{Smoothing linear filters}
\end{figure}

\begin{figure}[h!]
    \centering
    \includegraphics[scale=0.32]{src/img/dasgal10.png}
    \caption{Smoothing Linear Filters}
\end{figure}